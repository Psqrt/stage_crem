\documentclass{article}

\usepackage[utf8]{inputenc}
\usepackage[T1]{fontenc}
\usepackage[francais]{babel}
\usepackage[top=2cm, bottom=2cm, left=2cm, right=2cm]{geometry}

\usepackage{amsmath}
\usepackage{amsfonts}
\usepackage{amssymb}
\usepackage{gensymb}

\usepackage{tikz}
\usetikzlibrary{calc,angles,positioning,intersections,quotes,decorations.markings,backgrounds,patterns}
\usepackage{pgfplots}
\pgfplotsset{compat=1.11}
\usepackage{wrapfig}


\newtheorem{remark}{Remarque}

\newcommand{\floor}[1]{\lfloor #1 \rfloor}


\begin{document}

\pagestyle{empty}

\section*{Documentation de la fonction \texttt{N\_flows\_OD} (A. Gardahaut)}
Dans certains cas, nous souhaitons tracer plusieurs flux entre paires de points (Clubs pour nous), hors les fonctions habituellement utilisées en R et Leaflet ne permettent pas de générer plusieurs flows distincts, en conséquence, ils se superposent.

Nous souhaitions donc résoudre ce problème.

L'idée est basée sur le constat suivant, nous pouvons visualiser les N flux entre une même paire de points si ils sont tous de cambrures différentes.

 \begin{wrapfigure}{l}{12cm}
 	\begin{tikzpicture}
\begin{axis}[width=5in,axis equal image,
    axis lines = middle,
    xmin=-5,xmax=5,
    ymin=-1.5,ymax=5,
    restrict y to domain=-1.5:5,
    xtick={\empty},ytick={\empty},
    axis line style={-latex}, %latex-latex
    xlabel=$x$,ylabel=$y$,
    xlabel style={at={(ticklabel* cs:1)},anchor=north west},
    ylabel style={at={(ticklabel* cs:1)},anchor=south west}
]
\coordinate (origin) at (0,0);
\end{axis}

\begin{scope}[shift={(origin)}]
\coordinate (A) at (0,1);
\coordinate (B) at (3,0);
\coordinate (C) at ({5*cos(77.5)},{5*sin(77.5)});
\coordinate (D) at (-3,2);
\coordinate (O) at (0,0);
\draw (A) -- (B) -- (C) -- cycle;
\draw (A) -- (D) --(C);
\draw (B) -- (O);


\node[blue, below left] at (A){$A$};
\node[blue, below] at (B){$B$};
\node[blue,above left] at (C) {$C$};
\node[blue,above left] at (D) {$D$};
\node[blue,below left] at (O){$O$};


% Angles
\draw[draw=blue] (A) ++(77.5:0.4) arc (77.5:-20:0.4)
  node[midway,above right,inner sep=2pt,font={\footnotesize}]{$\theta$};

\draw[draw=blue] (B) ++(0:-0.9) arc (22:0:-0.9)
  node[midway,left,inner sep=5pt,font={\footnotesize}]{$\beta$};
	
	
\end{scope}
\end{tikzpicture}
\end{wrapfigure}

La cambrure étant déterminée par un angle $\alpha \in [0\degree,90\degree]$, avec:
\[ \alpha= \widehat{ABC} \]

L'idée pour générer les flux entre B et D est d'interpoler $n$ points entre B et D via le polynôme de Lagrange $P_{A}(x^{*})$ associé aux 3 points B, D et A, milieu de B et D dans le repère $(x^*, y^*)$ formé respctivement par (BD) et son perpendiculaire passant par B, une fois ce repère recentré en $(0,0)$ (afin d'utiliser les repères d'Euler).\\
On prendra autant de valeurs de $\alpha$ que de flux voulus.

On recentre d'après le point (B ou D) dont l'abscisse est la plus proche de $(0)$.

\[ P_{A}(x^{*})=\Delta \frac{(x^{*}-x_{B}^{*})(x^{*}-x_{D}^{*})}{(x_{A}^{*}-x_{B}^{*})(x_{A}^{*}-x_{D}^{*})} \]
Avec :
\[ \Delta= \sin(\alpha) \frac{d(B,D)}{2} \]

\begin{remark}
Pour éviter d'avoir un $\beta>90$, on change l'orientation du vecteur si nécessaire (c'est-à-dire si $\overrightarrow{BD}=(a,b), a<0).$
\end{remark}
Ensuite on calcule  :
\begin{center}
\[ (\cos(\beta),\sin(\beta))= \left(\frac{<\overrightarrow{BD},(1,0)>}{\| \overrightarrow{BD} \|},\frac{(1,0) \wedge \overrightarrow{BD}}{\| \overrightarrow{BD} \|}\right) \]
\end{center}
On calcule les coordonnées des points B, D, A dans $(x^{*},y^{*})$, on subdivise l'intervalle $[x_B^*,x_D^*]$ en $n$ points $(x_i)_{i \in [1:n]}$ et on calcule les :
$P_A(x_i^*)$ associés.

Le retour dans la base $(x,y)$ se fait via:
\begin{align*}
 \vec{x}& = \cos(\beta) x^* -\sin(\beta) y^*  \\
    \vec{y}& = \sin(\beta)x^* + \cos(\beta) y^* 
\end{align*}

Pour N pair, on distribue les flux $j$ et $j+1$ selon $\pm \alpha$ puis on augmente de $\alpha$ pour la prochaine paire de flux.
\begin{center}
\[ \alpha_j=  \floor{i/2} (-\alpha)^{j} \ \ \text{j} \in [2,N] \]
\end{center}
On prend par défaut $\alpha=10\degree$.

Pour le cas impair, on initialise le premier flux comme étant le segment passant par les 2 points (c'est-à-dire si $\alpha=0\degree$), puis pour les autres, on procède comme dans le cas pair.
\end{document}